\documentclass{article}
\title{Shell Interpreter With C Syntax}
\author{Zeyad Hany Moustafa ElShafei
      \\120210033
      \\CSE SEC 1
      \\Group 6
      \\Mon 11:15}
\date{March 2024 - May 2024}

\begin{document}
\maketitle

\section*{Project Overview:}
This project aims to develop an improved version of the BASH (Bourne-Again Shell) 
interpreter and command line utilities for Linux. The goal is to develop a more
humanly readable syntax for the end user in the terminal, making the process of scripting
and writing commands easier and more straightforward by using a more familiar and 
commonly used syntax, in this case; the C syntax.

\section*{Objectives:}

\begin{enumerate}

\item \textbf{Using The GNU Shell Utilities: } The shell includes most of the GNU BASH utilities,
including the file navigation utilities, and the find utilities.

\item \textbf{Improved Command Syntax: } Command syntax that is more readable and more intuitive to
the end user in the terminal, making the process of the remembering commands and typing them much
easier.

\item \textbf{Improved Scripting Syntax: } Scripting syntax that is less conflicting and draws a
more visual distinction between variables, strings, and environment variables.

\item \textbf{Add-Ons and Custom Commands: } A file where the user can define specific commands and 
their custom flags.

\end{enumerate}

\section*{Potential Challenges:}

\begin{enumerate}

\item \textbf{Creating A Runtime Environment: } The interpreter runs in a dedicated runtime environment that executes the 
commands and scripts.  

\item \textbf{Possible Conflicts In C Syntax And The Shell: } Conflicts in syntax between the C and the shell
syntax, which requires careful resolution and certain characters to be defined. 

\item \textbf{Integrating GNU Shell Utilities: } The GNU shell utilities integration requires some modifiaction to
the source code of GNU-utils.

\end{enumerate}

\section*{Application Areas:} 

\begin{enumerate}

\item \textbf{Linux Distributions: } Helping linux users to have a more amiable terminal environment where the 
shell is easier to use.

\item \textbf{Safer Usage: } Preventing obscure scripting mistakes from happening such as the Steam incident where
the installation script removed the root directory of a linux user. This is through better syntax decalartion 

\end{enumerate}

\section*{Expected Outcome: } 

A functioning shell with a custom syntax that includes the most of the BASH utilities and can be used by the average
linux user to perform main terminal tasks.

\end{document}





